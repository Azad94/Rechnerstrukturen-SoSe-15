\documentclass[11pt,a4paper]{article}
\usepackage{ucs}
\usepackage[utf8x]{inputenc}
\usepackage[T1]{fontenc}
\usepackage{ngerman}
\usepackage{amsmath,amssymb,amstext}
\usepackage{tikz}
\title{Rechnerstrukturen Blatt 1}
\author{Sven Marquardt}
\date{\today}
\input{kvmacros}
\begin{document}

Aufgabe 2\\
2.1)\\
Boolesche Funktion
f: $B^4$->$B^6$ \\
f($x_3,x_2,x_1,x_0$)=($y_5,y_4,y_3,y_2,y_1,y_0$),mit\\
$y_5 = x_3 \wedge x_2 \wedge x_1 \wedge x_0$\\
$y_4=(x_3 \wedge x_1) \vee(x_3 \wedge x_2)$\\
$y_3=(x_3 \wedge x_0) \vee (x_3 \wedge x_1) \vee (x_3 \wedge x_2)$\\
$y_2=(x_2 \wedge x_1 \wedge x_0 ) \vee x_3$\\
$y_1 = \lbrace x_3 \vee x_2 \wedge x_1 \rbrace $\\
$y_0=x_3 \vee x_2 \vee x_1 \vee x_0$\\ \\
2.2)\\
Die Gültigkeitsbereiche für alle fi von yi \\
ON($f_0$)= $\lbrace 0001,0010,0011,0100,0101,0110,0111,1000,1001,1010,1011,1100 \\
           1101,1110,1111 \rbrace$ \\
ON($f_1$)= $ \lbrace 0110,0111,1000,1001,1010,1011,1100,1101,1110,1111 \rbrace $ \\
ON($f_2$)= $ \lbrace 0111,1000,1001,1010,1011,1100,1101,1110,1111 \rbrace $ \\
ON($f_3$)= $ \lbrace 1001,1010,1011,1100,1101,1110,1111 \rbrace $\\
ON($f_4$)= $ \lbrace 1010,1011,1100,1101,1110,1111 \rbrace $ \\
ON($f_5$)= $ \lbrace 1111 \rbrace $ \\ \\
2.3)\\
\karnaughmap{4}{}{{$x_3$}{$x_2$}{$x_1$}{$x_0$}}{0111111111111111}{}\\
$y_0$=$\lbrace x_3 \vee x_2 \vee x_1 \vee x_0 \rbrace $\\ \\ \\ \\ \\ \\ \\ \\ \\ \\ \\
$l_0=\lbrace \\
\neg x_3 \wedge x_2 \wedge x_1 \wedge \neg x_0 ,\\
\neg x_3 \wedge x_2 \wedge x_1 \wedge x_0,\\
x_3 \wedge \neg x_2 \wedge \neg x_1 \wedge \neg x_0,\\
x_3 \wedge \neg x_2 \wedge \neg x_1 \wedge x_0,\\
x_3 \wedge \neg x_2 \wedge x_1 \wedge \neg x_0, \\
x_3 \wedge \neg x_2 \wedge x_1 \wedge x_0, \\
x_3 \wedge x_2 \wedge \neg x_1 \wedge \neg x_0,\\
x_3 \wedge x_2 \wedge \neg x_1 \wedge x_0,\\
x_3 \wedge x_2 \wedge x_1 \wedge \neg x_0, \\
x_3 \wedge x_2 \wedge x_1 \wedge x_0 \rbrace$\\ \\
$l_1$=$ \lbrace \\
\neg x_3 \wedge x_2 \wedge x_1,\\
x_2 \wedge x_1 \wedge x_0,\\
x_3 \wedge \neg x_2 \wedge \neg x_1,\\
x_3 \wedge \neg x_2 \wedge x_0,\\
x_3 \wedge \neg x_2 \wedge x_1,\\
x_3 \wedge x_1 \wedge x_0,\\
x_3 \wedge x_2 \wedge \neg x_1,\\
x_3 \wedge x_2 \wedge x_0,\\
x_3 \wedge x_2 \wedge x_1 \rbrace$\\ \\
$l_2$=$\lbrace 
x_2 \wedge x_1,
x_3 \wedge \neg x_2,
x_3 \wedge x_0,
x_3 \wedge x_1,
x_3 \wedge x_2 \rbrace$ \\
$p_2$=$\lbrace
x_2 \wedge x_1 \wedge x_0,
x_3 \wedge x_1 \wedge x_0 \rbrace $\\ \\
$l_3$=$\lbrace x_3 \rbrace$\\ 
$p_3$=$\lbrace x_2 \wedge x_1, x_3 \wedge x_0, x_3 \wedge x_1 \rbrace$\\
\begin{tabular}{l | c | c | c | c | c | c | c | c | c | c | c}
f&0110&0111&1000&1001&1010&1011&1100&1101&1110&1111 \\ \hline
$x_2 \wedge x_1 \wedge x_0$&0&1&0&0&0&0&0&0&0&1\\
$x_3 \wedge x_1 \wedge x_0$&0&0&0&0&0&1&0&0&0&1\\
$x_2 \wedge x_1$&1&1&0&0&0&0&0&0&1&1\\
$x_3 \wedge x_0$&0&0&0&1&0&1&0&1&0&1\\
$x_3 \wedge x_1$&0&0&0&0&1&1&0&0&1&1\\
$x_3$&0&0&1&1&1&1&1&1&1&1\\
\end{tabular}\\ \\
$x_3$ Dominiert die Zeile $x_3 \wedge x_1$, $ x_3 \wedge x_0$ und $x_3 \wedge x_1 \wedge x_0$. Deshalb können wir diese beiden Zeilen auslassen. Außerdem hat $x_3$ in den Spalten 1000 und 1100 die einzige 1. Wir können also auch diese Spalten raus lassen.Zeile $x_2 \wedge x_1$  hat in Spalte 0110 als einziger eine 1 also kann auch diese Spalte gekürzt werden.$x_2 \wedge x_1$ Dominiert auch die Zeile $x_2 \wedge x_1 \wedge x_0$.Übrig bleibt nur noch $x_3 \wedge x_1 \wedge x_0$ als Rest. Daraus folgt folgendes Minimalpolynom für $y_1$.
$y_1 = \lbrace x_3 \vee x_2 \wedge x_1 \rbrace $\\  \\
$l_0= \lbrace \\
\neg x_3 \wedge x_2 \wedge x_1 \wedge x_0,\\
x_3 \wedge \neg x_2 \wedge \neg x_1 \wedge \neg x_0,\\
x_3 \wedge \neg x_2 \wedge \neg x_1 \wedge x_0,\\
x_3 \wedge \neg x_2 \wedge x_1 \wedge \neg x_0, \\
x_3 \wedge \neg x_2 \wedge x_1 \wedge x_0, \\
x_3 \wedge x_2 \wedge \neg x_1 \wedge \neg x_0,\\
x_3 \wedge x_2 \wedge \neg x_1 \wedge x_0,\\
x_3 \wedge x_2 \wedge x_1 \wedge \neg x_0, \\
x_3 \wedge x_2 \wedge x_1 \wedge x_0 \rbrace$\\ \\
$l_1= \lbrace\\ 
x_2 \wedge x_1 \wedge x_0,\\
x_3 \wedge \neg x_2 \wedge \neg x_1,\\
x_3 \wedge \neg x_2 \wedge x_0,\\
x_3 \wedge \neg x_2 \wedge x_1,\\
x_3 \wedge x_1 \wedge x_0,\\
x_3 \wedge x_2 \wedge \neg x_1,\\
x_3 \wedge x_2 \wedge x_0,\\
x_3 \wedge x_2 \wedge x_1 \rbrace$\\ \\
$l_2=\lbrace 
x_2 \wedge x_1,
x_3 \wedge \neg x_2,
x_3 \wedge x_0,
x_3 \wedge x_1,
x_3 \wedge x_2 \rbrace$\\ 
$p_2= \lbrace x_2 \wedge x_1 \wedge x_0, x_3 \wedge x_1 \wedge x_0 \rbrace$\\ \\
$l_3$=$\lbrace x_3 \rbrace$\\
$p_3$=$\lbrace x_2 \wedge x_1, x_3 \wedge x_0, x_3 \wedge x_1 \rbrace$\\
\begin{tabular}{l | c | c | c | c | c | c | c | c | c | c }
f&0111&1000&1001&1010&1011&1100&1101&1110&1111 \\ \hline
$x_2 \wedge x_1 \wedge x_0$&1&0&0&0&0&0&0&0&1\\
$x_3 \wedge x_1 \wedge x_0$&0&0&0&0&1&0&0&0&1\\
$x_2 \wedge x_1$&1&0&0&0&0&0&0&1&1\\
$x_3 \wedge x_0$&0&0&1&0&1&0&1&0&1\\
$x_3 \wedge x_1$&0&0&0&1&1&0&0&1&1\\
$x_3$&0&1&1&1&1&1&1&1&1\\
\end{tabular}\\ \\
Durch Zeilendominanz bleiben nur $x_3$ und die $x_2 \wedge x_1$ übrig.\\
$y_2= \lbrace x_3 \vee x_2 \wedge x_1 \rbrace$\\ \\
$l_0= \lbrace
x_3 \wedge \neg x_2 \wedge \neg x_1 \wedge x_0,\\
x_3 \wedge \neg x_2 \wedge x_1 \wedge \neg x_0, \\
x_3 \wedge \neg x_2 \wedge x_1 \wedge x_0, \\
x_3 \wedge x_2 \wedge \neg x_1 \wedge \neg x_0,\\
x_3 \wedge x_2 \wedge \neg x_1 \wedge x_0,\\
x_3 \wedge x_2 \wedge x_1 \wedge \neg x_0, \\
x_3 \wedge x_2 \wedge x_1 \wedge x_0 \rbrace$\\ \\
$l_1= \lbrace 
x_3 \wedge \neg x_2 \wedge x_0,\\
x_3 \wedge \neg x_2 \wedge x_1,\\
x_3 \wedge x_1 \wedge x_0,\\
x_3 \wedge x_2 \wedge \neg x_1,\\
x_3 \wedge x_2 \wedge x_0,\\
x_3 \wedge x_2 \wedge x_1 \rbrace$\\ \\
$l_2= \lbrace 
x_3 \wedge x_0,
x_3 \wedge x_1,
x_3 \wedge x_2 \rbrace$\\ 
$p_2= \lbrace x_3 \wedge x_1 \wedge x_0 \rbrace$\\
$p_3= \lbrace 
x_3 \wedge x_0,
x_3 \wedge x_1,
x_3 \wedge x_2 \rbrace$\\
\begin{tabular}{l  | c | c | c | c | c | c | c | c }
f&1001&1010&1011&1100&1101&1110&1111 \\ \hline
$x_3 \wedge x_1 \wedge x_0$&0&0&1&0&0&0&1\\
$x_3 \wedge x_0$&1&0&1&0&1&0&1\\
$x_3 \wedge x_1$&0&1&1&0&0&1&1\\
$x_3 \wedge x_2$&0&0&1&1&1&1&1\\
\end{tabular}\\ \\
Durch Spaltendominanz kommen wir auf.\\
$y_3=\lbrace x_3 \wedge x_0 \vee x_3 \wedge x_1 \vee x_3 \wedge x_2 \rbrace$\\ \\
$l_0= \lbrace \\
x_3 \wedge \neg x_2 \wedge x_1 \wedge \neg x_0, \\
x_3 \wedge \neg x_2 \wedge x_1 \wedge x_0, \\
x_3 \wedge x_2 \wedge \neg x_1 \wedge \neg x_0,\\
x_3 \wedge x_2 \wedge \neg x_1 \wedge x_0,\\
x_3 \wedge x_2 \wedge x_1 \wedge \neg x_0, \\
x_3 \wedge x_2 \wedge x_1 \wedge x_0 \rbrace$\\ \\
$l_1= \lbrace 
x_3 \wedge \neg x_2 \wedge x_1,
x_3 \wedge x_1 \wedge x_0,
x_3 \wedge x_2 \wedge \neg x_1,
x_3 \wedge x_2 \wedge x_0,
x_3 \wedge x_2 \wedge x_1 \rbrace$\\ \\
$l_2= \lbrace 
x_3 \wedge x_1,
x_3 \wedge x_2 \rbrace$\\ 
$p_2= \lbrace x_3 \wedge x_1 \wedge x_0,x_3 \wedge x_2 \wedge x_0 \rbrace$\\
$p_3= \lbrace 
x_3 \wedge x_1,
x_3 \wedge x_2 \rbrace$\\
\begin{tabular}{l   | c | c | c | c | c | c | c }
f&1010&1011&1100&1101&1110&1111 \\ \hline
$x_3 \wedge x_1 \wedge x_0$&0&1&0&0&0&1\\
$x_3 \wedge x_2 \wedge x_0$&0&0&0&1&0&1\\
$x_3 \wedge x_1$&1&1&0&0&1&1\\
$x_3 \wedge x_2$&0&1&1&1&1&1\\
\end{tabular}\\
Durch Spaltendominanz kommen wir auf  $y_4= \lbrace x_3 \wedge x_1 \vee x_3 \wedge x_2 \rbrace$\\ 
Da $y_5$ sich nicht weiter vereinfachen lässt, ist das Minimalpolynom gleich der Normalform. 
$y_5= \lbrace x_3 \wedge x_2 \wedge x_1 \wedge x_0 \rbrace $\\ \\

Aufgabe 3 \\
3.1)\\
Partielle boolesche Funktion  \\
f: D-> $B^3$, mit $D= \lbrace 010,011,101,111 \rbrace$ und f$( x_2,x_1,x_0)=(y_2,y_1,y_0)$ ergibt\\
$y_2 = \lbrace \neg x_2 \wedge x_0 \vee \neg x_1 \rbrace $\\
$y_1 = \lbrace \neg x_0 \vee x_2 \rbrace $\\
$y_0 = \lbrace \neg x_2 \vee \neg x_1 \rbrace $\\ \\
3.2)\\
Die Wahrheitsmenge der fi von yi \\
$ON(f_2)=\lbrace 0111,101 \rbrace $\\
$ON(f_1)=\lbrace 010,101,111 \rbrace $\\
$ON(f_0)= \lbrace 010,011,101 \rbrace $\\
und deren Don't care Bereiche \\
$DC(f_2,f_1,f_0)= \lbrace 000,001,100,110 \rbrace $\\ \\ 
3.3)\\
$y_2$\\
\karnaughmap{3}{}{{$x_2$}{$x_1$}{$x_0$}}{**01*1*0}{}
$y_2 = \lbrace \neg x_2 \wedge x_0 \vee \neg x_1 \rbrace $\\ \\
$y_1$\\
\karnaughmap{3}{}{{$x_2$}{$x_1$}{$x_0$}}{**10*1*1}{}
$y_1 = \lbrace \neg x_0 \vee x_2 \rbrace $\\ \\
$y_0$\\
\karnaughmap{3}{}{{$x_2$}{$x_1$}{$x_0$}}{**11*1*0}{}
$y_0 = \lbrace \neg x_2 \vee \neg x_1 \rbrace $\\ \\

\end{document}

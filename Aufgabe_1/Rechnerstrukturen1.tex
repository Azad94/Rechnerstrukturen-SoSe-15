\documentclass[11pt,a4paper]{article}
\usepackage{ucs}
\usepackage[utf8x]{inputenc}
\usepackage[T1]{fontenc}
\usepackage{ngerman}
\usepackage{amsmath,amssymb,amstext}
\usepackage{tikz}
\title{Rechnerstrukturen Blatt 1}
\author{Sven Marquardt}
\date{\today}
\input{kvmacros}
\begin{document}


1-1 e)
\\
\begin{tabular}{c | c | c | c | c}
$x_0$ & $x_1$ & $x_2$& $x_3$ & y \\ \hline
0&0&0&0&0\\
0&0&0&1&0\\ \hline
0&0&1&0&0\\ \hline
0&0&1&1&1\\ \hline
0&1&0&0&0\\
0&1&0&1&0\\
0&1&1&0&0\\
0&1&1&1&1\\ \hline
0&0&0&0&0\\
0&0&0&1&0\\
1&0&1&0&0\\
1&0&1&1&1\\ \hline
1&1&0&0&0\\
1&1&0&1&1\\
1&1&1&0&0\\
1&1&1&1&0\\
\end{tabular}
\\
\\
Das Schaltnetz soll uns alle Primzahlen im 4 Bit Bereich anzeigen. Der Fehler ist, das die 2 (ON(f)=
$\lbrace 0100 \rbrace$)
nicht als Primzahl erkannt wird. \\ \\
1-2\\
c.i)\\
Durch das Assoziativgesetz  koennen wir die Klammern bei gleichem booleschem Ausdruck weglassen.
\begin{equation}
\label{Assoziativ1}
( a \wedge b) \wedge (c \wedge d) = (a \wedge b \wedge c) \wedge d=a \wedge b \wedge c \wedge d 
\end{equation}
\begin{equation}
( a \vee b)\vee (c \vee d)= (a \vee b \vee c) \vee d= a \vee b \vee c \vee d
\end{equation}
\\
c.ii)\\

\begin{equation}
\label{boolescher Ausdruck}
(( x_2 \wedge x_1 ) \wedge x_0 ) \vee x_2~Tiefe~3
\end{equation}
\begin{equation}
\label{boolesche Funktion}
((AND3(x_2,x_1,x_0))\vee x_2~Tiefe~2
\end{equation}
Einen booleschen Ausdruck ( \ref{boolescher Ausdruck} ) kann man nicht mit einem erweiterten booleschen Ausdruck  ( \ref{boolesche Funktion} ) vergleichen. \\ \\
d)\\
Der boolesche Ausdruck der Tabelle ausgeschrieben\\
$
y=( \neg x_3 \wedge \neg x_2 \wedge x_1 \wedge x_0) \vee  ( \neg x_3 \wedge x_2 \wedge \neg x_1 \wedge x_0 ) \vee  ( \neg x_3 \wedge x_2 \wedge x_1 \wedge x_0  ) \vee \\
( x_3 \wedge \neg x_2 \wedge x_1 \wedge x_0 ) \vee ( x_3 \wedge \neg x_2 \wedge \neg x_1 \wedge x_0 )
$
\\ \\
e)\\
Die korrekte boolesche Funktion \\
$
y=( \neg x_3 \wedge \neg x_2 \wedge x_1 \wedge \neg x_0 ) \vee ( \neg x_3 \wedge \neg x_2 \wedge x_1 \wedge x_0) \vee ( \neg x_3 \wedge x_2 \wedge \neg x_1 \wedge x_0) \vee \\ ( \neg x_3 \wedge x_2 \wedge x_1 \wedge x_0) \vee ( x_3 \wedge \neg x_2 \wedge x_1 \wedge x_0) \vee (x_3 \wedge x_2 \wedge \neg x_1 \wedge x_0)$\\
\\
1-4\\
a)\\
\begin{tabular}{c | c | c | c | c}
$x_3$&$x_2$&$x_1$&$x_0$&y\\ \hline
0&0&0&0&0\\
0&0&0&1&0\\
0&0&1&0&0\\
0&0&1&1&1\\ \hline
0&1&0&0&0\\
0&1&0&1&1\\
0&1&1&0&1\\
0&1&1&1&0\\ \hline
1&0&0&0&0\\
1&0&0&1&1\\
1&0&1&0&1\\
1&0&1&1&0\\ \hline
1&1&0&0&1\\
1&1&0&1&0\\
1&1&1&0&0\\
1&1&1&1&0\\
\end{tabular}\\ \\
Die Tabelle zeigt wie in der Aufgabe beschrieben eine totale boolesche Funktion mit dem Hamminggewicht von zwei\\ \\
KDNF \\
$\label{KDNF:KDNF}
y=(\neg x_3 \wedge \neg x_2 \wedge x_1 \wedge x_0)\vee ( \neg x_3 \wedge x_2 \wedge \neg x_1 \wedge  x_0) \vee ( \neg x_3 \wedge x_2 \wedge x_1 \wedge \neg x_0) \vee \\ ( x_3 \wedge \neg x_2 \wedge \neg x_1 \wedge x_0) \vee (x_3 \wedge \neg x_2 \wedge x_1 \wedge \neg x_0) \vee ( x_3 \wedge x_2 \wedge \neg x_1 \wedge \neg x_0) $\\
\\
Die Erfüllbarkeitsmenge der booleschen Funktion. 
\begin{displaymath}
ON(f)=\lbrace0011,0101,0110,1001,1010,1100\rbrace\\
\end{displaymath}
\karnaughmap{4}{}{{$x_3$}{$x_2$}{$x_1$}{$x_0$}}{0001011001101000}{}\\
Dadurch das wir in unserem KV-Diagramm keine gruppen bilden können, kann man diese Schaltung nicht weiter vereinfachen. Somit ist die minimalste Darstellung dieser Schaltung gleich der DNF.\\
\\ \\
d)
Vereinfacht wird die Schaltung durch das zusammenführen gleicher ausdrücke wie schon in (\ref{Assoziativ1}) gezeigt.
\end{document}